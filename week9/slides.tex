\documentclass{beamer}

\usetheme{Frankfurt}

\title{Workshop 8 -- Structs}
\author{COMP20005 Engineering Computation}
\institute{The University of Melbourne}

\begin{document}

\renewcommand{\tt}[1]{\texttt{#1}}

\begin{frame}
    \titlepage
\end{frame}

\begin{frame}{Contact Info}
    \begin{itemize}
        \item Matt Signorini
        \item Email: \tt{msignorini@unimelb.edu.au}
        \item Also LMS discussion forum for help
        \item My workshop solutions available at \url{https://github.com/pkill-9/comp20005-workshops/}
    \end{itemize}
\end{frame}

\begin{frame}{Today's Workshop}
    \begin{block}{Agenda}
        By the end of this class, you should:
        \begin{itemize}
            \item Understand the difference between a struct definition
                and a variable declaration.
            \item Be able to design simple structs to encapsulate a group
                of related variables.
        \end{itemize}
    \end{block}
\end{frame}

\section{Exercise: 8.2}
\stepcounter{subsection}

\begin{frame}{8.2 -- Vectors}
    Define a structure \tt{vector\_t} that stores (x,y) coordinates.
    Then write a function \tt{float distance(vector\_t p1, vector\_t p2)}
    that returns the distance between the two points, using Pythagoras'
    formula:
    \begin{equation}
        \sqrt{(x_1 - x_2)^2 + (y_1 - y_2)^2}
    \end{equation}
\end{frame}

\section{Exercise: 8.5}
\stepcounter{subsection}

\begin{frame}{8.5 -- Complex numbers}
    Define a structure for storing complex numbers, each with a real and
    imaginary part, both of type \tt{double}.

    Then write functions to add and multiply complex numbers using your
    struct.
\end{frame}

\section{Lab Exercises}
\stepcounter{subsection}

\begin{frame}{Exercises}
    \begin{block}{Problems}
        \begin{itemize}
            \item[8.3] Define a polygon struct, and function to calculate
                perimeter.
            \item[8.4] Using the same polygon struct, write a function to
                calculate area.
        \end{itemize}
    \end{block}
\end{frame}

%
%  ____________________________________
% / As of next week, passwords will be \
% \ entered in Morse code.             /
%  ------------------------------------
%   \            .    .     .   
%    \      .  . .     `  ,     
%     \    .; .  : .' :  :  : . 
%      \   i..`: i` i.i.,i  i . 
%       \   `,--.|i |i|ii|ii|i: 
%            UooU\.'@@@@@@`.||' 
%            \__/(@@@@@@@@@@)'  
%                 (@@@@@@@@)    
%                 `YY~~~~YY'    
%                  ||    ||     
%


\end{document}

% vim: ft=tex ts=4 sw=4 lbr et
