\documentclass{beamer}

\usepackage{minted}

\usetheme{Frankfurt}

\title{Workshop 2 -- Making Choices}
\author{COMP20005 Engineering Computation}
\institute{The University of Melbourne}

\begin{document}

\begin{frame}
    \titlepage
\end{frame}

\begin{frame}{Today's Workshop}
    \begin{block}{Agenda}
        In today's class, you should learn:
        \begin{itemize}
            \item How to use selection statements in C.
            \item Common gotcha's with logic operators.
            \item Writing comments to document your code.
        \end{itemize}
    \end{block}
\end{frame}

\section{Exercise: 3.2}
\stepcounter{subsection}

\begin{frame}[fragile]{3.2 a)}
    \begin{minted}{c}
    i=3; j=4;
    if (i<j && j<6) {
        i = i+j;
    } else {
        j = i+j;
    }
    printf ("i=%d, j=%d\n", i, j);
    \end{minted}
\end{frame}

\begin{frame}[fragile]{3.2 b)}
    \begin{minted}{c}
    i=3; j=4; k=7;
    if ((i<j || j<k) && j<i) {
        i = i+1;
        if (i*i > k) {
            k = k+1;
        }
    } else {
        j = j+1;
        if (i*i > k) {
            k = k+2;
        }
    }
    printf ("i=%d, j=%d, k=%d\n", i, j, k);
    \end{minted}
\end{frame}

\begin{frame}[fragile]{3.2 c)}
    \begin{minted}{c}
        month = 7;
        if (month == 2) {
            days = 28;
        } else if (month == 4 || 6 || 9 || 11) {
            days = 30;
        } else {
            days = 31;
        }
        printf ("days=%d\n", days);
    \end{minted}
\end{frame}

\begin{frame}[fragile]{3.2 d)}
    \begin{minted}{c}
        x = 1; y = 2;
        if (x>y)
            printf ("x=%d, y=%d\n", x, y);
            x = x+1;
        if (x<y)
            printf ("x=%d, y=%d\n", x, y);
            y = y+2;
        printf ("x=%d, y=%d\n", x, y);
    \end{minted}
\end{frame}

\begin{frame}[fragile]{3.2 e)}
    \begin{minted}{c}
        x = 1; y = 2;
        if (x>y); {
            printf ("x=%d, y=%d\n", x, y);
            x = x+1;
        }
        if (x<y); {
            printf ("x=%d, y=%d\n", x, y);
            y = y+2;
        }
        printf ("x=%d, y=%d\n", x, y);
    \end{minted}
\end{frame}

\begin{frame}[fragile]{3.2 f)}
    \begin{minted}{c}
        x = 0; y = 0;
        if (y<x) {
            printf ("y is smaller than x\n");
        } else if (y=x) {
            printf ("x and y are equal\n");
        } else {
            printf ("y is greater\n");
        }
    \end{minted}
\end{frame}

% I know you're only reading this for the ASCII art...
%
%
%  ________________________________________
% /  The master programmer moves from      \
% | program to program without fear. No    |
% | change in management can harm him. He  |
% | will not be fired, even if the project |
% | is canceled. Why is this? He is filled |
% | with the Tao.                          |
% |                                        |
% | -- Geoffrey James, "The Tao of         |
% \ Programming"                           /
%  ----------------------------------------
%      \
%       \
%        ("`-'  '-/") .___..--' ' "`-._
%          ` *_ *  )    `-.   (      ) .`-.__. `)
%          (_Y_.) ' ._   )   `._` ;  `` -. .-'
%       _.. `--'_..-_/   /--' _ .' ,4
%    ( i l ),-''  ( l i),'  ( ( ! .-'    
%
%
% 
% Remember to comment your code, especially for projects, because other
% people will need to read it. You don't need to include any pretty ASCII
% pictures like this though. It's ok, you can stop panicing now :)

\end{document}

% vim: ft=tex ts=4 sw=4 lbr et
