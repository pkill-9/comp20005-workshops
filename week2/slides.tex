\documentclass{beamer}

\usetheme{Frankfurt}

\title{Workshop 1 -- Introduction to C}
\author{COMP20005 Engineering Computation}
\institute{The University of Melbourne}

\begin{document}

\begin{frame}{Challenge}
    You are standing at an unmarked intersection. One way goes to Sydney, 
    and the other way to Melbourne.
    \begin{itemize}
        \item People from Sydney always lie.
        \item People from Melbourne always tell the truth.
    \end{itemize}
    A person from one of these places (you don't know which) is standing at
    the intersection. Is there a question you can ask to find out the way to
    Melbourne?
\end{frame}

\begin{frame}
    \titlepage
\end{frame}

\section{Introductions}
% this stepcounter command is needed to include the dots in the frankfurt
% theme. It is a nasty kludge, but at least it makes the presentation
% more beautiful. Like sausages, you don't wanna know how it's made.
\stepcounter{subsection}

\begin{frame}{}
    \begin{itemize}
        \item Tell everyone your name, and your most creative suggestion to
            destroy a computer and/or lose data.
        \item Are you doing the subject because you want to or have to?
        \item Have you programmed in any other languages before? Which one(s)?
        \item What is one piece of technology you wish existed?\footnote{\url{http://xkcd.com/1382/}}
    \end{itemize}
\end{frame}

\begin{frame}{About Me}
    \begin{itemize}
        \item Matt Signorini
        \item Masters of Science (Computer Science)
    \end{itemize}
\end{frame}

\section{Basics of C Programming}
\stepcounter{subsection}

\begin{frame}{Agenda}
    \begin{block}{Outcomes}
        At the end of todays class, you should know how to:
        \begin{itemize}
            \item Write simple C programs using the jEdit text editor.
            \item Compile and run C programs using a command line.
            \item Use IO functions and arithmetic operators in C.
        \end{itemize}
    \end{block}
\end{frame}

\begin{frame}{Example: IO Basics}
    Write a program that asks the user to enter two numbers, read them in
    from the terminal, and then compute the sum, difference, product and
    quotient of the two numbers.
\end{frame}

\begin{frame}{Tasks for today}
    \begin{block}{Write, compile and execute}
        \begin{itemize}
            \item Exercise 1.2 (blue) 1.3 (yellow): Hello World program.
            \item Exercise 2.8 (blue) 2.6 (yellow): Fahrenheit to Celsius
                conversion.
        \end{itemize}
    \end{block}
\end{frame}

% And now some ASCII art, for those who are dedicated enough to read my
% LaTeX source code.
%
%
%  _________________________________________
% / Q: How many Californians does it take   \
% | to screw in a light bulb? A: Five. One  |
% | to screw in the light bulb and four to  |
% | share the                               |
% |                                         |
% | experience. (Actually, Californians     |
% | don't screw in                          |
% |                                         |
% | light bulbs, they screw in hot tubs.)   |
% |                                         |
% | Q: How many Oregonians does it take to  |
% | screw in a light bulb? A: Three. One to |
% | screw in the light bulb and two to fend |
% | off all                                 |
% |                                         |
% | those Californians trying to share the  |
% \ experience.                             /
%  -----------------------------------------
%      \
%       \
%              ,;;;;;;;,
%             ;;;;;;;;;;;,
%            ;;;;;'_____;'
%            ;;;(/))))|((\
%            _;;((((((|))))
%           / |_\\\\\\\\\\\\
%      .--~(  \ ~))))))))))))
%     /     \  `\-(((((((((((\\
%     |    | `\   ) |\       /|)
%      |    |  `. _/  \_____/ |
%       |    , `\~            /
%        |    \  \           /
%       | `.   `\|          /
%       |   ~-   `\        /
%        \____~._/~ -_,   (\
%         |-----|\   \    ';;
%        |      | :;;;'     \
%       |  /    |            |
%       |       |            |

\end{document}

% vim: ft=tex ts=4 sw=4 lbr et
