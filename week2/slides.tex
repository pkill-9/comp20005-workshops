\documentclass{beamer}

\usetheme{Frankfurt}

\title{Workshop 1 -- Introduction to C}
\author{COMP20005 Engineering Computation}
\institute{The University of Melbourne}

\begin{document}

\begin{frame}{Challenge}
    You are standing at an unmarked intersection. One way is the city of
    lies, and the other way is the city of truth.
    \begin{itemize}
        \item People from the city of lies always lie.
        \item People from the city of truth always tell the truth.
    \end{itemize}
    A person from one of these cities (you don't know which) is standing at
    the intersection. Is there a question you can ask to find out the way to
    the city of truth?\footnote{Pierce, Rod. (13 Sep 2013). ``City of Lies or Truth Puzzle''. Math Is Fun. Retrieved 4 Mar 2015 from http://www.mathsisfun.com/puzzles/city-of-lies-or-truth.html}
\end{frame}

\begin{frame}
    \titlepage
\end{frame}

\section{Introductions}
% this stepcounter command is needed to include the dots in the frankfurt
% theme. It is a nasty kludge, but at least it makes the presentation
% more beautiful. Like sausages, you don't wanna know how it's made.
\stepcounter{subsection}

\begin{frame}{Turn to the person next to you\ldots}
\end{frame}

\begin{frame}{Second Slide Name Here}
\end{frame}

\section{Basics of C Programming}
\stepcounter{subsection}

\begin{frame}{Example: IO Basics}
    Write a program that asks the user to enter two numbers, read them in
    from the terminal, and then compute the sum, difference, product and
    quotient of the two numbers.
\end{frame}

\begin{frame}{Tasks for today}
    \begin{block}{Write, compile and execute}
        \begin{itemize}
            \item Exercise 1.2 (blue) 1.3 (yellow): Hello World program.
            \item Exercise 2.8 (blue) 2.6 (yellow): Fahrenheit to Celsius
                conversion.
        \end{itemize}
    \end{block}
\end{frame}


\end{document}

% vim: ft=tex ts=4 sw=4 lbr et
