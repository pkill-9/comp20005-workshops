\documentclass{beamer}

\usetheme{Frankfurt}

\title{Workshop 10 -- Binary Numbers}
\author{COMP20005 Engineering Computation}
\institute{The University of Melbourne}

\begin{document}

\renewcommand{\tt}[1]{\texttt{#1}}

\begin{frame}
    \titlepage
\end{frame}

\begin{frame}{Contact Info}
    \begin{itemize}
        \item Matt Signorini
        \item Email: \tt{msignorini@unimelb.edu.au}
        \item Also LMS discussion forum for help
        \item My workshop solutions available at \url{https://github.com/pkill-9/comp20005-workshops/}
    \end{itemize}
\end{frame}

\begin{frame}{Today's Workshop}
    \begin{block}{Agenda}
        By the end of this class, you should:
        \begin{itemize}
            \item Be able to convert numbers between decimal and binary
                form for both two's complement integers and simple 
                floating point numbers.
            \item Be able to do simple arithmetic operations on numbers
                in binary form.
        \end{itemize}
    \end{block}
\end{frame}

\section{Exercises}
\stepcounter{subsection}

\begin{frame}{13.1}
    Suppose that a machine uses 6 bits to represent integers. Calculate
    the two's complement representations for each of these values:
    0, 4, 9, -1, -8 and -31. Verify that $19 - 8 = 11$
\end{frame}

\begin{frame}{13.2}
    Consider a 16 bit floating point representation that uses 1 bit for 
    the sign, 3 bits for the exponent and 12 bits for the mantissa. 
    Calculate the binary representations of these numbers: 2.0, -2.5, 
    7.875.

    You will need to convert the fractions from decimal to binary, for
    which you may find this website helpful: 
    \url{http://cs.furman.edu/digitaldomain/more/ch6/dec\_frac\_to\_bin.htm}
\end{frame}

\begin{frame}{13.3}
    Using the same 16 bit float representation, what are the largest and
    smallest values that can be represented, in terms of absolute value?
\end{frame}


\section{Lab Exercises}
\stepcounter{subsection}

\begin{frame}{Exercises}
    \begin{block}{Project 2}
        Continue working on your projects. If you have finished stage 5,
        try to further refine your work. Do you have some long functions?
        Could your approach to the problem be better described in your
        comments? (you \emph{do} have comments, right?)
    \end{block}
\end{frame}

%
%  ____________________________________
% / If God had a beard, he'd be a UNIX \
% \ programmer.                        /
%  ------------------------------------
%     \
%      \
%                                    .::!!!!!!!:.
%   .!!!!!:.                        .:!!!!!!!!!!!!
%   ~~~~!!!!!!.                 .:!!!!!!!!!UWWW$$$ 
%       :$$NWX!!:           .:!!!!!!XUWW$$$$$$$$$P 
%       $$$$$##WX!:      .<!!!!UW$$$$"  $$$$$$$$# 
%       $$$$$  $$$UX   :!!UW$$$$$$$$$   4$$$$$* 
%       ^$$$B  $$$$\     $$$$$$$$$$$$   d$$R" 
%         "*$bd$$$$      '*$$$$$$$$$$$o+#" 
%              """"          """"""" 
%


\end{document}

% vim: ft=tex ts=4 sw=4 lbr et
