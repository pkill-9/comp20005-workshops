\documentclass{beamer}

\usepackage{minted}

\usetheme{Frankfurt}

\title{Workshop 3 -- Loops}
\author{COMP20005 Engineering Computation}
\institute{The University of Melbourne}

\begin{document}

\begin{frame}
    \titlepage
\end{frame}

\begin{frame}{Contact Info}
    \begin{itemize}
        \item Matt Signorini
        \item Email: \texttt{msignorini@unimelb.edu.au}
        \item Also LMS discussion forum
    \end{itemize}
\end{frame}

\begin{frame}{Today's Workshop}
    \begin{block}{Agenda}
        In today's class, you should learn:
        \begin{itemize}
            \item How to construct different kinds of loops in C
            \item How to write a program that reads an arbitrary ammount
                of input
            \item How to redirect your program's input to come from a
                file
        \end{itemize}
    \end{block}
\end{frame}

\section{Exercise: 4.1}
\stepcounter{subsection}

\begin{frame}[fragile]{4.1 a)}
    \begin{minted}{c}
        for (i=0; i<20; i=i+3) {
            printf ("%2d\n", i);
        }
    \end{minted}
\end{frame}

\begin{frame}[fragile]{4.1 b)}
    \begin{minted}{c}
        for (i=1; i<2000000; i=2*i) {
            printf ("%7d\n", i);
        }
    \end{minted}
\end{frame}

\begin{frame}[fragile]{4.1 c)}
    \begin{minted}{c}
        sum = 0;
        for (i=1; i<10; i++) {
            sum = sum+i;
            printf ("S (%2d) = %2d\n", i, sum);
        }
    \end{minted}
\end{frame}

\begin{frame}[fragile]{4.1 d)}
    \begin{minted}{c}
        for (i=0; i<8; i++) {
            for (j=i+1; j<8; j+=3) {
                printf ("i=%d, j=%d\n", i, j);
            }
        }
    \end{minted}
\end{frame}

\begin{frame}[fragile]{4.1 e)}
    \begin{minted}{c}
        for (i=0; i<8; i++) {
            for (j=i+1; j<8; j+=3) {
                if (i+j==7) {
                    break;
                }
                printf ("i=%d, j=%d", i, j);
            }
        }
    \end{minted}
\end{frame}

\begin{frame}[fragile]{4.1 f)}
    \begin{minted}{c}
        j = 5;
        for (i=0; i<j; i++); {
            printf ("i=%d, j=%d\n", i, j);
        }
    \end{minted}
\end{frame}

\begin{frame}[fragile]{4.1 g)}
    \begin{minted}{c}
        j = 5;
        for (i=0; i<j; j++) {
            printf ("i=%d, j=%d\n", i, j);
        }
    \end{minted}
\end{frame}

\section{Exercise: 4.2}
\stepcounter{subsection}

\begin{frame}[fragile]{4.2}
    Consider the following general do while construct. How can we express
    this as a for loop?
    \begin{minted}{c}
        do {
            commands;
        } while (condition);
    \end{minted}
\end{frame}

\section{Lab Exercises}
\stepcounter{subsection}

\begin{frame}{Exercises}
    \begin{block}{Exercises}
        \begin{itemize}
            \item Exercise 4.5 -- printing a graph.
            \item Exercise 4.9 -- find the next largest prime number.
            \item Exercise 4.4 -- print a table of ASCII characters.
        \end{itemize}
    \end{block}
    \begin{itemize}
        \item To test your solution to 4.5, you could create a text file
            containing some numbers, and use input redirection.
        \item Your program should read a single number, print a line of
            stars, then move on to the next number. It does not matter if
            the user typed multiple numbers on a single line.
        \item Solutions available at \url{https://github.com/pkill-9/comp20005-workshops/}
    \end{itemize}
\end{frame}

%
%
%  _______________________________________
% / Connection reset by some moron with a \
% \ backhoe                               /
%  ---------------------------------------
%   \                                  ,+*^^*+___+++_
%    \                           ,*^^^^              )
%     \                       _+*                     ^**+_
%      \                    +^       _ _++*+_+++_,         )
%               _+^^*+_    (     ,+*^ ^          \+_        )
%              {       )  (    ,(    ,_+--+--,      ^)      ^\
%             { (@)    } f   ,(  ,+-^ __*_*_  ^^\_   ^\       )
%            {:;-/    (_+*-+^^^^^+*+*<_ _++_)_    )    )      /
%           ( /  (    (        ,___    ^*+_+* )   <    <      \
%            U _/     )    *--<  ) ^\-----++__)   )    )       )
%             (      )  _(^)^^))  )  )\^^^^^))^*+/    /       /
%           (      /  (_))_^)) )  )  ))^^^^^))^^^)__/     +^^
%          (     ,/    (^))^))  )  ) ))^^^^^^^))^^)       _)
%           *+__+*       (_))^)  ) ) ))^^^^^^))^^^^^)____*^
%           \             \_)^)_)) ))^^^^^^^^^^))^^^^)
%            (_             ^\__^^^^^^^^^^^^))^^^^^^^)
%              ^\___            ^\__^^^^^^))^^^^^^^^)\\
%                   ^^^^^\uuu/^^\uuu/^^^^\^\^\^\^\^\^\^\
%                      ___) >____) >___   ^\_\_\_\_\_\_\)
%                     ^^^//\\_^^//\\_^       ^(\_\_\_\)
%                       ^^^ ^^ ^^^ ^


\end{document}

% vim: ft=tex ts=4 sw=4 lbr et
