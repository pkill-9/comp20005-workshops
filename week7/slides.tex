\documentclass{beamer}

\usetheme{Frankfurt}

\title{Workshop 6 -- Arrays}
\author{COMP20005 Engineering Computation}
\institute{The University of Melbourne}

\begin{document}

\begin{frame}
    \titlepage
\end{frame}

\begin{frame}{Contact Info}
    \begin{itemize}
        \item Matt Signorini
        \item Email: \texttt{msignorini@unimelb.edu.au}
        \item Also LMS discussion forum
        \item My workshop solutions available at \url{https://github.com/pkill-9/comp20005-workshops/}
    \end{itemize}
\end{frame}

\begin{frame}{Today's Workshop}
    \begin{block}{Agenda}
        In today's class, you should learn:
        \begin{itemize}
        \end{itemize}
    \end{block}
\end{frame}

\section{Exercise: 7.1}
\stepcounter{subsection}

\begin{frame}{7.1}
    Write a function \tt{int all\_zero(int a[], int n)} which returns true
    if the the elements \tt{a[0]} to \tt{a[n - 1]} are all zero.
\end{frame}

\section{Exercise: 7.2}
\stepcounter{subsection}

\begin{frame}{7.2}
    Modify the bubblesort program from the book so that the array is sorted
    into decreasing order.
\end{frame}

\section{Exercise: 7.3}
\stepcounter{subsection}

\begin{frame}{7.3}
    Modify your bubblesort program even further so that once the array has
    been sorted, only distinct values remain in the array (with the length
    variable reduced accordingly).
\end{frame}

\section{Lab Exercises}
\stepcounter{subsection}

\begin{frame}{Exercises}
    \begin{block}{More project work}
        With your knowledge of arrays, you will be able to complete stage
        3 of the project, and further refine your implementation of stages
        1 and 2.
    \end{block}
\end{frame}

%
%  _________________________________________
% /  There was once a programmer who worked \
% | upon microprocessors. "Look at how well |
% | off I am here," he said to a mainframe  |
% | programmer who came to visit, "I have   |
% | my own operating system and file        |
% | storage device. I do not have to share  |
% | my resources with anyone. The software  |
% | is self-consistent and easy-to-use. Why |
% | do you not quit your present job and    |
% | join me here?"                          |
% |                                         |
% | The mainframe programmer then began to  |
% | describe his system to his friend,      |
% | saying: "The mainframe sits like an     |
% | ancient sage meditating in the midst of |
% | the data center. Its disk drives lie    |
% | end-to-end like a great ocean of        |
% | machinery. The software is as           |
% | multi-faceted as a diamond and as       |
% | convoluted as a primeval jungle. The    |
% | programs, each unique, move through the |
% | system like a swift-flowing river. That |
% | is why I am happy where I am."          |
% |                                         |
% | The microcomputer programmer, upon      |
% | hearing this, fell silent. But the two  |
% | programmers remained friends until the  |
% | end of their days.                      |
% |                                         |
% | -- Geoffrey James, "The Tao of          |
% \ Programming"                            /
%  -----------------------------------------
%     \                                  ___-------___
%      \                             _-~~             ~~-_
%       \                         _-~                    /~-_
%              /^\__/^\         /~  \                   /    \
%            /|  O|| O|        /      \_______________/        \
%           | |___||__|      /       /                \          \
%           |          \    /      /                    \          \
%           |   (_______) /______/                        \_________ \
%           |         / /         \                      /            \
%            \         \^\\         \                  /               \     /
%              \         ||           \______________/      _-_       //\__//
%                \       ||------_-~~-_ ------------- \ --/~   ~\    || __/
%                  ~-----||====/~     |==================|       |/~~~~~
%                   (_(__/  ./     /                    \_\      \.
%                          (_(___/                         \_____)_)
%


\end{document}

% vim: ft=tex ts=4 sw=4 lbr et
