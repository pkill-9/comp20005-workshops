\documentclass{beamer}

\usetheme{Frankfurt}

\title{Workshop 9 -- More Structs}
\author{COMP20005 Engineering Computation}
\institute{The University of Melbourne}

\begin{document}

\renewcommand{\tt}[1]{\texttt{#1}}

\begin{frame}
    \titlepage
\end{frame}

\begin{frame}{Contact Info}
    \begin{itemize}
        \item Matt Signorini
        \item Email: \tt{msignorini@unimelb.edu.au}
        \item Also LMS discussion forum for help
        \item My workshop solutions available at \url{https://github.com/pkill-9/comp20005-workshops/}
    \end{itemize}
\end{frame}

\begin{frame}{Today's Workshop}
    \begin{block}{Agenda}
        By the end of this class, you should:
        \begin{itemize}
            \item Understand how structs can be used in a program.
            \item Be able to define your own structs.
        \end{itemize}
    \end{block}
\end{frame}

\section{Exercise: 8.9}
\stepcounter{subsection}

\begin{frame}{8.9 -- Sorting structs}
    Look again at the declarations in figure 8.4 in the textbook.

    Write a function \tt{sort\_staff\_by\_number} that sorts an array of
    type \tt{staff\_t} in order of increasing \tt{employeenumber}.

    You may use any of the sorting algorithms that you encountered in
    chapter 7.
\end{frame}

\section{Lab Exercises}
\stepcounter{subsection}

\begin{frame}{Exercises}
    \begin{block}{Project 2}
    \end{block}
\end{frame}

%
%  _________________________________________
% / Even if you aren't in doubt, consider   \
% | the mental welfare of the person who    |
% | has to maintain the code after you, and |
% | who will probably put parens in the     |
% | wrong place. -- Larry Wall in the perl  |
% \ man page                                /
%  -----------------------------------------
%                \                    ^    /^
%                 \                  / \  // \
%                  \   |\___/|      /   \//  .\
%                   \  /O  O  \__  /    //  | \ \           *----*
%                     /     /  \/_/    //   |  \  \          \   |
%                     @___@`    \/_   //    |   \   \         \/\ \
%                    0/0/|       \/_ //     |    \    \         \  \
%                0/0/0/0/|        \///      |     \     \       |  |
%             0/0/0/0/0/_|_ /   (  //       |      \     _\     |  /
%          0/0/0/0/0/0/`/,_ _ _/  ) ; -.    |    _ _\.-~       /   /
%                      ,-}        _      *-.|.-~-.           .~    ~
%     \     \__/        `/\      /                 ~-. _ .-~      /
%      \____(oo)           *.   }            {                   /
%      (    (--)          .----~-.\        \-`                 .~
%      //__\\  \__ Ack!   ///.----..<        \             _ -~
%     //    \\               ///-._ _ _ _ _ _ _{^ - - - - ~
%


\end{document}

% vim: ft=tex ts=4 sw=4 lbr et
