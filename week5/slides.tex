\documentclass{beamer}

\usetheme{Frankfurt}

\title{Workshop 4 -- Abstraction with Functions}
\author{COMP20005 Engineering Computation}
\institute{The University of Melbourne}

\begin{document}

\begin{frame}
    \titlepage
\end{frame}

\begin{frame}{Contact Info}
    \begin{itemize}
        \item Matt Signorini
        \item Email: \texttt{msignorini@unimelb.edu.au}
        \item Also LMS discussion forum
        \item My workshop solutions available at \url{https://github.com/pkill-9/comp20005-workshops/}
    \end{itemize}
\end{frame}

\begin{frame}{Today's Workshop}
    \begin{block}{Agenda}
        In today's class, you should learn:
        \begin{itemize}
            \item How to write and interpret simple functions.
            \item How to express a problem in terms of functions.
        \end{itemize}
    \end{block}
\end{frame}

\section{Exercise: 5.1}
\stepcounter{subsection}

\begin{frame}{Max of two ints}
    Write a function \texttt{max\_2\_ints} which takes two int arguments
    and returns the larger one.
\end{frame}

\section{Exercise: 5.2}
\stepcounter{subsection}

\begin{frame}{Max of four ints}
    Write a function \texttt{max\_4\_ints} that returns the largest of it's
    four int arguments.
\end{frame}

\section{Exercise: 5.3}
\stepcounter{subsection}

\begin{frame}{Raise an int to a given power}
    Write a function that takes two ints, and returns the first argument
    raised to the power of the second argument. Note that C does not
    provide a power operator for int arithmetic in the way that Matlab does.
\end{frame}

\section{Lab Exercises}
\stepcounter{subsection}

\begin{frame}{Exercises}
    \begin{block}{Exercises}
        \begin{itemize}
            \item Exercise 5.5 -- perfect numbers
            \item Exercise 5.6 -- amicable pairs
        \end{itemize}
    \end{block}
\end{frame}

\begin{frame}{Perfect Numbers}
    \begin{itemize}
        \item You will need a loop that finds factors and adds them to a
            sum. Then compare the sum of the factors to the number.
        \item \emph{Hint:} Brute force is acceptable for finding the next
            perfect number. Can you use your \texttt{isperfect} function to
            help here?
    \end{itemize}
\end{frame}

\begin{frame}{Amicable Pairs}
    \begin{itemize}
        \item Consider two numbers a and b.
        \item Factors of a must add up to b \emph{and}
        \item Factors of b must add up to a.
    \end{itemize}
\end{frame}

%
%   _________________________________________
%  /  There was once a programmer who was    \
%  | attached to the court of the warlord of |
%  | Wu. The warlord asked the programmer:   |
%  | "Which is easier to design: an          |
%  | accounting package or an operating      |
%  | system?"                                |
%  |                                         |
%  | "An operating system," replied the      |
%  | programmer.                             |
%  |                                         |
%  | The warlord uttered an exclamation of   |
%  | disbelief. "Surely an accounting        |
%  | package is trivial next to the          |
%  | complexity of an operating system," he  |
%  | said.                                   |
%  |                                         |
%  | "Not so," said the programmer, "when    |
%  | designing an accounting package, the    |
%  | programmer operates as a mediator       |
%  | between people having different ideas:  |
%  | how it must operate, how its reports    |
%  | must appear, and how it must conform to |
%  | the tax laws. By contrast, an operating |
%  | system is not limited by outside        |
%  | appearances. When designing an          |
%  | operating system, the programmer seeks  |
%  | the simplest harmony between machine    |
%  | and ideas. This is why an operating     |
%  | system is easier to design."            |
%  |                                         |
%  | The warlord of Wu nodded and smiled.    |
%  | "That is all good and well, but which   |
%  | is easier to debug?"                    |
%  |                                         |
%  | The programmer made no reply.           |
%  |                                         |
%  | -- Geoffrey James, "The Tao of          |
%  \ Programming"                            /
%   -----------------------------------------
%         \ (__)
%           (oo)
%     /------\/
%    / |    ||
%   *  /\---/\
%      ~~   ~~
%


\end{document}

% vim: ft=tex ts=4 sw=4 lbr et
