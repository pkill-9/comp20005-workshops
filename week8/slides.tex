\documentclass{beamer}

\usetheme{Frankfurt}

\title{Workshop 7 -- More Arrays}
\author{COMP20005 Engineering Computation}
\institute{The University of Melbourne}

\begin{document}

\renewcommand{\tt}[1]{\texttt{#1}}

\begin{frame}
    \titlepage
\end{frame}

\begin{frame}{Contact Info}
    \begin{itemize}
        \item Matt Signorini
        \item Email: \tt{msignorini@unimelb.edu.au}
        \item Also LMS discussion forum for help
        \item My workshop solutions available at \url{https://github.com/pkill-9/comp20005-workshops/}
    \end{itemize}
\end{frame}

\begin{frame}{Today's Workshop}
    \begin{block}{Agenda}
        By the end of this class, you should:
        \begin{itemize}
            \item Have a solid understanding of arrays in C.
            \item Be able to use arrays to solve a problem.
        \end{itemize}
    \end{block}
\end{frame}

\section{Exercise: 7.6}
\stepcounter{subsection}

\begin{frame}{7.6}
    An alternative sorting algorithm is to locate the largest value in the
    array and swap it to the last position, then repeat the process with
    the values that have not been swapped. This algorithm is known as
    \emph{selection sort}

    Write a function \tt{void selection\_sort (int a [], int n)} that sorts
    the array \tt{a} using the above algorithm.
\end{frame}

\section{Exercise: 3.6}
\stepcounter{subsection}

\begin{frame}{3.6 Revisited}
    In the lab in week 3, you created a program to list the
    coins required to make up a specified value. You may recall hints that
    there was a much more elegant solution than one implemented solely with
    if constructs. How would you write this program differently, given your 
    new knowledge of arrays?
\end{frame}

\section{Lab Exercises}
\stepcounter{subsection}

\begin{frame}{Exercises}
    \begin{block}{Problems}
        \begin{itemize}
            \item[7.4] program to print a histogram of int values.
            \item[7.7] write a function that returns the most commonly
                occurring value in an array of ints.
            \item[7.15] write a function that tests if two words are
                anagrams (7.16 in the old textbook edition).
        \end{itemize}
    \end{block}
\end{frame}

%
%  _________________________________________
% / The primary purpose of the DATA         \
% | statement is to give names to           |
% | constants; instead of referring to pi   |
% | as 3.141592653589793 at every           |
% | appearance, the variable PI can be      |
% | given that value with a DATA statement  |
% | and used instead of the longer form of  |
% | the constant. This also simplifies      |
% | modifying the program, should the value |
% | of pi change.                           |
% |                                         |
% \ -- FORTRAN manual for Xerox Computers   /
%  -----------------------------------------
%  \                   .,
%    \         .      .TR   d'
%      \      k,l    .R.b  .t .Je
%        \   .P q.   a|.b .f .Z%		
%            .b .h  .E` # J: 2`     .
%       .,.a .E  ,L.M'  ?:b `| ..J9!`.,
%        q,.h.M`   `..,   ..,""` ..2"`
%        .M, J8`   `:       `   3;
%    .    Jk              ...,   `^7"90c.
%     j,  ,!     .7"'`j,.|   .n.   ...
%    j, 7'     .r`     4:      L   `...
%   ..,m.      J`    ..,|..    J`  7TWi
%   ..JJ,.:    %    oo      ,. ....,
%     .,E      3     7`g.M:    P  41
%    JT7"'      O.   .J,;     ``  V"7N.
%    G.           ""Q+  .Zu.,!`      Z`
%    .9.. .         J&..J!       .  ,:
%       7"9a                    JM"!
%          .5J.     ..        ..F`
%             78a..   `    ..2'
%                 J9Ksaw0"'
%                .EJ?A...a.
%                q...g...gi
%               .m...qa..,y:
%               .HQFNB&...mm
%                ,Z|,m.a.,dp
%             .,?f` ,E?:"^7b
%             `A| . .F^^7'^4,
%              3.MMMMMMMMMMMQzna,
%          ...f"A.JdT     J:    Jp,
%           `JNa..........A....af`
%                `^^^^^'`
%


\end{document}

% vim: ft=tex ts=4 sw=4 lbr et
